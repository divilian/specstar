\documentclass{beamer}
\usepackage{amsmath}
\usepackage{amssymb}
\usepackage{amsfonts}
\usepackage{commath}
%\usepackage{graphics}
%\usepackage{curves}
\usepackage{tikz}
%\usetikzlibrary{backgrounds}
%\usetikzlibrary{snakes} 
%\usepackage{amsmath}
%\usepackage{amssymb}
%\usepackage{amsfonts}
%\usepackage{graphics}
%\usepackage{curves}
%\usepackage{tikz}
%\usetikzlibrary{backgrounds}
%\usetikzlibrary{snakes}
%\usepackage{setspace}
%\doublespacing
%\usepackage{lscape}
%\usepackage{booktabs}
%\usepackage{longtable}
\usepackage[normalem]{ulem}
\usepackage{hyperref}
\usepackage{url}
\usepackage{color}
\usepackage{bibentry}
\usepackage{graphicx}

%\nobibliography*
\bibliographystyle{plain}
\graphicspath{ {./figures/} }
\mode<presentation>
{
  \usetheme{default}      % or try Darmstadt, Madrid, Warsaw, ...
  \usecolortheme{default} % or try albatross, beaver, crane, ...
  \usefonttheme{default}  % or try serif, structurebold, ...
  \setbeamertemplate{navigation symbols}{}
  \setbeamertemplate{caption}[numbered]
} 

\usepackage[english]{babel}
\usepackage[utf8x]{inputenc}

\beamertemplatenavigationsymbolsempty

\setbeamerfont{page number in head/foot}{size=\small}
\setbeamertemplate{footline}[frame number]

\title[Your Short Title]{Wealth dynamics in the presence of network structure and primitive cooperation}
\author[shortname]{Rajesh Venkatachalapathy \inst{1} \and Stephen Davies
\inst{2} \\ \and William Nehrboss \inst{3}}
\institute[shortinst]{\inst{1} Systems Science Graduate Program \\Portland State University \\Portland, Oregon \and \inst{2} Department of Computer Science\\University of Mary Washington\\Fredericksburg, Virginia \and \inst{3} Lake Anna Homeschool\\Bumpass, Virginia}

%\author{Rajesh Venkatachalapathy}
%\institute{Systems Science Graduate Program \\Portland State University \\Portland, Oregon}
%
%
%
%\author{Stephen Davies}
%\institute{Systems Science Graduate Program \\Portland State University \\Portland, Oregon}
%\date{Friday 15th February 2019}




\begin{document}

\begin{frame}
\titlepage
\end{frame}


\begin{frame}[t]
\frametitle{Outline}
\begin{itemize}
\item Motivation and questions
\item Model
\item Experiments 
\item Sample simulation runs
\item Findings 
\item Future work 
\end{itemize}
\end{frame}

\begin{frame}[t]
\frametitle{Motivation}
\begin{itemize}
\item Friesen-Mudigonda (FM) model 
\end{itemize}
Questions
\begin{itemize}
\item What is the social network analog of the FM model?
\item What is the mathematical analog of ABM-based FM model? 
\item Do social structures mediate wealth accumulation? 
\item Does cooperation lead to more wealth? (Or to less?)
\item Does cooperation improve chances of survival under resource scarcity?
\end{itemize}
\end{frame}

\begin{frame}[t]
\frametitle{Model}
FM model
\begin{itemize}
\item Agent-based model 
\item Resources are distributed in a grid world 
\item Agents interact as they forage for resources in \textit{this} grid world
\item  \textit{Wealthy enough} agents who encounter one another decide to pool their resources  
\end{itemize}
\bigskip
Our model 
\begin{itemize}
\item Stochastic dynamical system 
\item Resources are provided to each agent in a \textit{noisy manner}
\item Agents interact with neighbors in a static social network
\item \textit{Wealthy enough} agents who encounter one another decide to pool their resources 
\end{itemize}
\end{frame}

\begin{frame}[t]
\frametitle{Model}

Dynamics
\begin{equation}\label{ddm}
\dif x(t) = v\dif t + \sqrt{2D} \dif w
\end{equation}

\begin{itemize}
\item $v$ is the wealth growth rate (the difference of the income and metabolic
rate of the agent) and $D$ is the intensity of the Brownian (white noise)
process $w$. 
\item $x(t)$ is the state of the particle (wealth) at time $t$. 
\item The dynamics can be started at any initial point $x_0 > 0$ (diffusion on
a \textit{semi-infinite interval}).
\end{itemize}
Structure
\begin{itemize}
\item Erd\H{o}s-R\'{e}nyi network (ER) models
\end{itemize}
\end{frame}

\begin{frame}[t]
\frametitle{Vocabulary}
\begin{itemize}
\item an \textbf{isolate}
\item a ``\textbf{proto}'' a.k.a proto-institution
\item ``\textbf{starvation shock}''
\item \textbf{lifespan}
\item $\lambda$ parameter (for ER model)
\item \texttt{julia}
\end{itemize}
\end{frame}

\begin{frame}[t]
\frametitle{Experiments}
\bigskip
There are three \textbf{stages} in our experiments
\smallskip
\begin{itemize}
\item \textbf{Stage 1} is the phase of the dynamical model \textit{before}
which any agents have formed protos.
\item \textbf{Stage 2} is the phase \textit{during} which agents are forming protos.
\item \textbf{Stage 3} is the phase \textit{after} which all (non-isolate)
agents have joined a proto. \textbf{Starvation} commences.
\end{itemize}
\end{frame}

\begin{frame}[t]
\frametitle{Demo}
\bigskip
\bigskip
\bigskip
\bigskip
\begin{center}
\huge
Sample simulation runs
\end{center}
\end{frame}

\begin{frame}[t]
\frametitle{Experimental results}
\begin{figure}[hb]
\centering
\includegraphics[scale=.4]{figures/giniVsLambda.png}
\caption{The average Gini coefficient of effective wealth (computed pre-Stage 3) for various values of the ER $\lambda$ connectivity parameter, and with both low-noise and high-noise income. 500 agents were used in each simulation. The color band represents a bootstrapped 95\% confidence interval.}
\label{fig:giniVsLambda}
\end{figure}
\end{frame}

\begin{frame}[t]
\frametitle{Experimental results}
\begin{figure}[ht]
\centering
\includegraphics[scale=.4]{figures/avgLifespanLambda_025.png}
\caption{Life expectancy comparison between isolates (non-proto members) and
non-isolates (proto members) for different values of $\lambda$ and $\sigma^2$.
The \texttt{salary} parameter was set to 20, so the x-axis ranges from a nearly
constant agent income to a scenario when the noise is as high as the average.}
\label{fig:avgLifespanLambda025}
\end{figure}
\end{frame}

\begin{frame}[t]
\frametitle{Experimental results}
\begin{figure}[ht]
\centering
\includegraphics[scale=.4]{figures/avgLifespanLambda2.png}
\caption{Life expectancy comparison between isolates (non-proto members) and
non-isolates (proto members) for different values of $\lambda$ and $\sigma^2$.
The \texttt{salary} parameter was set to 20, so the x-axis ranges from a nearly
constant agent income to a scenario when the noise is as high as the average.}
\label{fig:avgLifespanLambda2}
\end{figure}
\end{frame}

\begin{frame}[t]
\frametitle{Takeaways}
%% need to link the figures here 
\begin{itemize}
\itemsep1em
\item Higher environmental noise produces more inequality (Fig.~1)
\item Denser connectivity produces \textit{less} inequality (Fig.~1)
\item Higher environmental noise gives proto members a longer average lifespan (Figs.~2 and 3) 
    \begin{itemize}
    \item ...and this effect is exaggerated for more densely connected populations
    \end{itemize}
\end{itemize}

\end{frame}

\begin{frame}[t]
\frametitle{Questions}
\begin{itemize}
\item What are the mechanisms underlying the gap between mean life expectancy
of proto members and isolates?
\item Do protos of different sizes differ in the mean life expectancy
and wealth? 
\item Can we derive the results of these simulations from the defining SDS?
\item Do other baseline network models (like \textit{scale-free} and
\textit{small-world}) induce similar systematic differences between
proto members and isolates?
\end{itemize}
\end{frame}

\begin{frame}[t]
\frametitle{Conclusion}
\begin{itemize}
\item These are very preliminary results 
\item Repeating these simulations on other baseline network models might provide more clarity
\item Simulations with SDS models used in macroeconomics will help empirically ground the models  
\item These models might be amenable to mean field analysis 
\end{itemize}
\bigskip
More generally
\begin{itemize}
\item Our models are an useful addition to the variety of dynamical system
models on networks.
\item Despite their simplicity, they exhibit interesting dynamics and are useful to study the two-way interaction between structure and dynamics.  
\end{itemize}
\end{frame}

\end{document}
