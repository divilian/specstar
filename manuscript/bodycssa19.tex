\section{Introduction}
In recent years, the concerns and debates regarding wealth inequality and socioeconomic mobility have been one of the few unifying issues dominating the extremely polarized public spheres of the Global North. While economic and political inequality used to be discussed in heterodox economics circles, the contentious discussions on the topic within mainstream economics since the publication of Piketty's book~\cite{piketty2017capital} suggest a lack of consensus about basic foundational questions like the origins and persistence of economic inequality. Not surprisingly, traditional theories and tools of macro and micro economics are now being diagnosed for their limitations. Simultaneously, insights from related disciplines, along with novel models for scientific inquiry not usually associated with traditional econometrics, are being taken more seriously. This work contributes to this effort by integrating substantive ideas from anthropology, economic sociology~\cite{granovetter2017society} and urban sociology~\cite{sampson2012great} and using modeling approaches from computational social science~\cite{hedstrom2018}, computational economics~\cite{tesfatsion} and analytical sociology~\cite{hedstrom2011oxford} to explore the origins of inequality in a simple model of wealth dynamics, with social structures and a primitive form of cooperation .    

The current work originated in our attempts to incorporate and tease out the effects of social structures in simple models of wealth dynamics in the presence of environmental stochasticity and a simple form of resource pooling. We are interested in the interaction between structure and dynamics, their role in the production and persistence of inequality, and how they influence the system's robustness to scarcity shocks. We answer the former question using Gini coefficients and the latter using survival time type analysis. As we discuss in section 3, while Gini was found to be of limited value, survival time analysis produced more insights on both questions of inequality and robustness. 

The model presented here was inspired by our search for analogs of the grid world agent-based model (ABM) of Friesen and Mudi\-gonda~\cite{srimil} where foraging agents that pool their resources were shown to on-average outperform non-pooling agents. This model used foraging to mix the population and create opportunities for interaction, and, when certain conditions were met, resource sharing. Our model achieves agent interaction by postulating a static network structure which partially mixes the agents. Of primary interest is whether network effects alone can generate and sustain differences in economic outcomes of otherwise homogeneous agents.  

The Friesen and Mudigonda model drew its inspiration from historical sociology, in Katz's influential study of middle of 19th century Hamilton, Canada~\cite{katz2013people}. Retaining this original motivation, we draw additional inspiration from economic sociology in the work of Granovetter~\cite{granovetter2005}; urban sociology, in the work of Sampson~\cite{sampson2002} and others; and in anthropology, in the work on cooperation in small to medium scale societies~\cite{avner1994,white2011kinship}, and others. The geographic and economic scale of the systems and the nature of social actors and time scale of interest in these different disciplineary approaches are all very different from the ones used to develop models of representative agents in macroeconomics~\cite{benhabib2018}, making the similarity between macroeconomic wealth dynamics models and our models not comparable without further justification. Elaborating on the interplay of conceptual and methodological ideas among these disciplines is beyond the scope of this article. Instead, we anchor this work in economic sociology and revisit the insights from the above disciplines in the article's concluding section.

The important role played by social structures in determining economic outcomes of individuals in a society is not in doubt~\cite{granovetter2005, jackson_rev2017}. Still, in the absence of a unifying foundation for sociology and economics, the full impact of this two-way interpenetration of economic and social structures is demonstrated only on a case-by-case basis. The language of social and economic networks affords a first principle integration by simplifying the non-trivial concept of social and economic structure~\cite{martin_lee} to only dyadic (binary) relations among actors.\footnote{The ``social structure'' concept~\cite{martin_lee} is more general than the social network specifically, despite the widely held belief in computational social science that they are synonyms.}

Models of wealth accumulation dynamics in both macro and micro economics typically assume the presence of markets and equilibria. Hence, such models are ill-suited to the study of collective phenomena at an intermediate level. Alternative explanations of aggregate phenomena that match the expressiveness of economic models are required. Analytical sociology~\cite{ch1as_hdbk}, with its emphasis on explanation of collective emergent phenomena using mathematically formulated social mechanisms~\cite{ch2as_hdbk,ch11as_hdbk} anchored at the individual level, is an ideal candidate for this purpose. It is particularly powerful when combined with computational simulation, because when mathematically formulated models reach even a modest level of complexity, they often become analytically intractable. Simulation \textit{in silico} can yield approximate results for these more complex scenarios, which supplement the exact results reached by analytically tractable models. 

Models constructed here, like those used in scientific inquiry in general, serve a specific purpose. In this work, we construct simple toy models to reproduce certain aspects of non-trivial wealth inequality distributions in the presence and absence of primitive forms of cooperation, clearly delineating the role of network structure in generating wealth inequality. We make no suggestions that these models \textit{explain} the phenomena of interest; we are only interested in constructing the simplest possible models, with no detailed empirical grounding, but with the potential to generate heterogeneous wealth outcomes reminiscent of ones obtained using more sophisticated models. As we discuss later, these simple mechanisms are limited in their ability to stabilize wealth and inequality. Despite this, the model still sheds light on the true social and economic mechanisms underlying the genesis and persistence of economic inequality, and network structure alone can produce large differences in economic outcomes. 


\label{giniKindaSucks}
The phenomenon under scrutiny is the emergence and evolution of economic inequality in societies with non-trivial social structures that enable economic interactions and cooperation mechanisms. The model we use to answer questions surrounding this phenomenon must possess a dynamic model of wealth accumulation, a model of social structure, and a suitably useful measure of inequality. The dynamics are modeled as Brownian-noise-driven linear dynamical systems, here with constant growth rate; the structure is modeled by networks, here with random graphs\footnote{In this paper (as elsewhere) the terms \textbf{graph} and \textbf{network} are exact synonyms.}; the measures of inequality used are Gini coefficients, and survival time distributions of agents in response to lack of resources. Just as with macroeconomic models, measures of wealth inequality like Gini coefficients are not so well suited for the non-market based wealth dynamics. We discuss this more in section 4. 


Although both the dynamical system and the network model are quite well understood, the precise interplay of structure and dynamics produces interesting emergent wealth distributions and robustness to scarcity. To the best of our knowledge, this specific combination of network structure and social dynamics has not been discussed in the computational social science (CSS) or mathematical sociology literature, and we consider this the primary contribution of this work. 

Apart from enabling interactions between social actors, social structures like institutions also shape the form of cooperation and coordination mechanisms. The institutions can take the form of economic institutions, like banks and cooperatives; or the form of norms, like resource-sharing practices in societies. The Friesen and Mudigonda model~\cite{srimil} consists of a simple resource pooling arrangement where aggregates of agents pool their excess wealth in a common institution called a ``proto-institution'' (henceforth, ``\textbf{proto}'') agreeing to provide this saved resource to individual agents in times of need. The model of resource pooling used in this work is identical to this model. 

The analysis to be presented in later sections focuses only on homogeneous agents with simple drift-diffusion dynamics on Erd\H{o}s-R\'{e}nyi network models (ER); space constraints unfortunately prevent us from repeating the analysis on other standard \textit{textbook} networks like scale-free and small-world networks. We discuss empirical evidence for the role of social network structures in the concluding section of this paper, motivating the need for more expressive social network models. 


In the next section, we discuss the mathematical formulation of the model. In section 3, we present the analysis of our simulation experiments, summarize key findings, and discuss why we chose the Julia programming language~\cite{Julia-2017} for our implementation. In section 5, we discuss the limitations of our simple models, extensions to dynamics and networks more expressive than the ones presented here and planned future work. 

Before discussing our model in greater mathematical detail, we discuss the model qualitatively, contrasting it with more familiar modeling approaches. The model presented here has much in common with models used in social-reality-inspired models in statistical physics~\cite{redner2001guide}, dynamic process models in network science~\cite{newman2018networks}, computational social science, and agent-based models~\cite{abm_review}; however, our modeling philosophy is somewhere at the interface of agent-based computational economics (ACE)~\cite{tesfatsion} and analytical sociology (AS)~\cite{ch1as_hdbk,ch2as_hdbk}. We acknowledge ACE's aspiration to develop bottom-up models of economic systems at all scales, but restrain from its enthusiastic use of complex but well-calibrated detailed models of markets and agents~\cite{tesfatsion2017}. We adhere to AS's focus on social mechanisms in explaining social phenomena, but instead rely on simplified mechanisms with few parameters~\cite{ch11as_hdbk} with the specific goal of extracting insights from \textit{stylized} models.       

More specifically, from statistical physics, we borrow the dynamics: diffusion models and associated first-passage time techniques; from network science, we borrow the structural aspects: Erd\H{o}s-R\'{e}nyi network (ER) models; and from non-equilibrium statistical physics, CSS and ACE, we borrow a form of cooperation: the concept of coalescence, institution and coordination. 


\section{Model}
The model of cooperative wealth accumulation constructed here is best thought of as a stochastic interacting particle system infused with economic sociological semantics. The particle evolves according to a one-dimensional diffusion process with constant drift and is driven by Brownian noise with a boundary condition at the origin corresponding to particle absorption. After crossing a pre-determined threshold in state space, particles above the threshold follow a protocol and coalesce together. In an ensemble of otherwise identical particles, a given particle may coalesce with a subset of other particles. In grid world ABMs, agents interact with other agents by moving around this world. The precise movement protocol encodes how the agents interact (mix) among themselves. In the social network setting, this mixing characteristic is encoded via a graph. Both the conjoined particles and individual particles die upon crossing the origin. This model of interacting diffusing particles can be provided with substantive semantics as follows. 

The one-dimensional state space of the particle is identified with the wealth of a social actor (agent). We consider a homogeneous population where all agents are required burn their wealth at a constant specified rate in order to survive. In addition, the agents all gain wealth at a constant rate. The resource draining rate and the resource gaining rate are additive and constitute the drift of the diffusion process. The environmental contingency is modeled by a Brownian noise of a specified intensity. Agents can coalesce to form a cooperative unit deciding to pool their resources and their environmental contingencies into a single unit which we call a proto-institution (proto), if they cross a specific wealth threshold. The mixing characteristic of this ensemble is the social network (here ER model).          

The questions of interest to us, emergence and persistence of inequality and robustness to scarcity, in the presence of social structure and uncertain environmental conditions, map onto questions about the stochastic dynamical system (SDS). Inequality can be quantified using Gini coefficients of the particle ensemble's state space. Robustness to scarcity can be measured using survival time analysis for the particles to reach the absorbing boundary at the origin by turning off the appropriate drift terms in the dynamics. 

\subsection{Mathematical formulation}
The discussion of SDS closely follows ~\cite{redner2001guide}\footnote{The use of such analysis for studying wealth dynamics was presented by one of the authors at CSSA18 ~\cite{rv}. Unlike the finite interval dynamics used there, the dynamics here take place on a semi-infinite line.}.

The SDS can be defined through a stochastic differential equation of the form 

\begin{equation}\label{ddm}
dx(t) = vdt + \sqrt{2D} dw
\end{equation}

\noindent where $v$ is the wealth growth rate (the difference of the income and metabolic rate of the agent) and $D$ is the intensity of the Brownian process (white noise process) $w$. $x(t)$ is the state of the particle (wealth) at time $t$. The dynamics can be started at any initial point $x_0 > 0$. Since simulations make use of discrete versions of these equations, a slightly different notation is used\footnote{The discrete time step $\Delta t=1$. $w$ in the discrete setting is a Gaussian distributed random variable $\mathcal{N}(0,\sigma^2) = \mathcal{N}(0,D\Delta t)$.}.

While equation (\ref{ddm}), in its discretized form, is the basis for simulations, other formulations are used~\cite{redner2001guide,rv}\footnote{These formulations make use measure theoretical probability to convert SDEs to partial differential equations known as Fokker-Planck equations.  The provide numerical and closed form estimates of probability density, survival time probabilities and other quantities of importance.}.

For a single particle, the probability of survival $S(t)$ up to time $t$ and the probability of reaching the origin can be calculated and are functions of $x_0$, $v$ and $D$. For example, the expected probability of a particle reaching the origin ($\mathcal{E}(x_0)$), when starting at $x_0$ is given by 

\begin{equation}\label{e}
\mathcal{E}(x_0) = \begin{cases}
e^{-v x_0/D} \text{, } & \text{ for } v > 0 \\
1 \text{, } & \text{ for } v \leq 0 \\
\end{cases}
\end{equation}  

\noindent That is, there is always a non-zero probability of reaching the origin, even when there is a constant positive drift away from the origin; and, the return to origin is certain for negative or zero drift.  Similarly, survival probability, $\mathcal{S}(t)$ is given by 

\begin{equation}\label{s}
\mathcal{S}(t) = \begin{cases}
1 - e^{-v x_0/D}\text{, } & \text{ } v >0 \\
\sqrt{\frac{4D}{\pi v^2 t} e^{-v^2 t/4 D}}\text{, } & \text{ } v \leq 0 \\
\end{cases}
\end{equation}

Since the particles are independent, they are independent dynamical systems; proto formation is a higher-level construct imposed on this system and is a constraint on what states are considered viable and which are not. Consider two particles $x_1$ and  $x_2$ that have crossed $x_{\textrm{thresh}}$, the threshold for coalescence. After the coalescence event, the state (wealth) of individual particles $x_1(t)$ and  $x_2(t)$ is not relevant for survivability; only the aggregate wealth of the coalescent (proto) determines whether the particles in the coalescent survive. As long as $x_1(t) + x_2(t) >0$, both particles survive. Effectively, the proto is a two-dimensional SDS. Since the wealth of the proto ($p_{12}$) is additive, the aggregate wealth $p_{12}(t) = x_1(t) + x_2(t) $. The aggregate dynamical variable satisfies a SDE\footnote{The result follows from the additivity properties of white noise.}\footnote{The two pictures: the particle perspective and the proto perspective, are equivalent. While it is easier to mathematically analyze the system in the proto perspective, the individual wealth of the particles carries meaning; it is just not useful for studying survival of the proto or the particles within it.}.  

\begin{equation}\label{pddm}
dx(t) = 2vdt + \sqrt{2(2D)} dw
\end{equation}

As one can see, only the coefficient of drift and diffusion in equations (\ref{ddm}) and (\ref{pddm}) are different. So, the corresponding expressions for $\mathcal{E}(x_0)$ and $\mathcal{S}(t)$ are suitably scaled. This has implications for both Gini coefficient calculations and survival analysis calculations. 

For such ensemble of particles, the primary driver of difference in paths (life histories) and wealth is the Brownian noise intensity $D$; greater the $D$ leads greater the diversity (and hence Gini). However, Gini is a measure that is dependent on absolute magnitude. So, if the drift ($v$) is large, then the variation generated by
$D$ gets washed out by $v$\footnote{Preliminary investigations suggest that for simple non-network ensembles, Gini either stays close to $0$ or $1$. We suspect that this is because of the constant wealth growth rate used in our models, Gini is a partially useful measure. This is unlike in macroeconomic models where exponential growth rate gives rise to stable non-trivial Gini coefficients}. On the other hand, the mathematical form of expressions for $\mathcal{S}(t)$ suggest a clear dependence on $D$ which separates the population of particles that are not in a coalescent and the population of particles that are in one. Particles reaching the absorbing state (death of the agent) without being part of a coalescent are called isolates\footnote{In the context of a system with non-trivial network structure, an \textbf{isolate} agent is one with no graph neighbors.}; the non-isolates, the particles that are in a coalescent when they reach the absorbing state are called protos. 

As the equations (\ref{e}) and (\ref{s}) show, all agents, irrespective of their starting initial condition and luck, have a non-zero probability of dying. The paper that inspired this work ~\cite{srimil} studied inequality dynamics in the context of extreme scarcity in their foraging world. One can mimic this scarcity by simply ``turning off'' the income at some point, starving the agents, and leaving the agents to all inevitably die. The differential rate of death between the isolates and the non-isolates then becomes an important quantity. In the absence of tractable mathematical solutions, simulation experiments which probe this expected difference in survival rates help illuminate the role of protos. 

We first evolve the system under favorable environmental conditions with steady salary, during what we define as ``Stages 1 and 2.'' \textbf{Stage 1} is the phase of the dynamic model \textit{before} which any agents have formed protos (since their wealth has not yet reached $x_{\textrm{thresh}}$), and \textbf{Stage 2} is the phase \textit{during which} agents are forming protos. Once all non-isolate agents have joined a proto, \textbf{Stage 3} commences, at which point we cut off the salary for all agents, starving them and leaving them susceptible to the white noise process. The effective dynamics are different in each stage, but the form of the equations remains the same. 

To break the homogeneity of this particle ensemble, we propose to let the particles interact with only a random subset of other populations. In other words, we embed the particles on a network (graph); particles interact (form protos) with only their neighbors. In this work, we focus on the simplest \textit{textbook} example of network: Erd\H{o}s-R\'{e}nyi network model~\cite{newman2018networks}. ER models have only one parameter ($\lambda$), which is the \textit{average expected degree} of any given node. This determines the density of connections in the network, and in this setting determines the expected number of particles available for a particle to coalesce with and form a proto. Commonsense reasoning suggests that a larger $\lambda$ parameter may help increase the average lifetimes of individual agents by enabling proto formation by increasing the likelihood of contact with other particles. 

Just as equations (\ref{e}) and (\ref{s}) offer insight into the role of dynamics, the key feature of ER models is the high probability of the formation of a ``giant component'' at $\lambda \ge 1$. When a giant component exists, a finite fraction of the population is connected. Networks with this property also tend to have cliques of very high order, and other useful properties that bear upon proto formation.

As its network properties are well-understood, the network's wealth dynamics allow us to understand more thoroughly the interaction between dynamics and structure, and their role in global behavior. Still, a full analysis of this model remains to be completed and is part of a forthcoming work involving other more expressive network models. As we note subsequently, even this simple setting offers interesting insights regarding the role of network structure in determining important outcomes for the agent. 


\subsection{Implementation}


The model is realized in a discrete-time Julia simulation program, in which each agent is represented as a mutable struct (a dedicated area of memory whose contents change over time) that maintains its state. An agent's state includes its current amount of wealth and which proto (if any) it is a member of. Additionally, using the LightGraphs package~\cite{LightGraphs-2017}, each agent is associated with a node of a randomly generated ER graph. Protos are also represented as mutable structs, each of which contains a list of member agent IDs and a current wealth balance.

The simulation\footnote{All code for this work is available at \texttt{https://github.com/WheezePuppet/specstar}.} proceeds as follows. For configurable parameters
$N > 0$,
$\texttt{init\_max} > 0$,
$\texttt{salary} > 0$,
$m > 0$ (metabolic rate),
$x_{\textrm{thresh}} > 0$ (the ``proto threshold''),
$\sigma^2 \ge 0$ and
$0 \le \lambda \le 1$:

\begin{enumerate}
\itemsep.1em
\item Create $N$ agents, each with a random initial wealth (uniformly distributed from 0 to \texttt{init\_max}), and related to one another as per a random ER network with parameter $\lambda$.
\item \textbf{Stages 1 and 2:} Repeat until all non-isolate agents are members of a proto:
    \begin{enumerate}
    \itemsep.1em
    \item Each agent $A$ whose current wealth $\ge x_{\textrm{thresh}}$, and who is not currently a member of a proto, chooses at random one of its graph neighbors (call it $B$) whose wealth \textit{also} exceeds $x_{\textrm{thresh}}$. (If there are no such neighbors, proceed to the next agent.) If $B$ is already in a proto, have $A$ join $B$'s proto. If not, have $A$ and $B$ form a new proto.
    \item Each agent gains an amount of wealth equal to $(\texttt{salary} - m + \epsilon)$, where $\epsilon \sim \mathcal{N}(0,\sigma^2)$ (white noise).\footnote{Note that this ``gain'' could be negative, in which case the agent, and possibly its proto, may be subject to death exactly as in step (3).}
    \item Each agent \textit{that is in a proto} donates all its wealth in excess of $x_{\textrm{thresh}}$ to that proto's balance. (Agents not in a proto maintain their current wealth.)
    \end{enumerate}
    \item \textbf{Stage 3 (starvation):} Repeat until all agents are dead:
        \begin{enumerate}
        \itemsep.1em
        \item Each agent \textit{loses} an amount of wealth equal to $(m + \epsilon), \epsilon \sim \mathcal{N}(0,\sigma^2)$.
        \item If an agent's wealth would drop below zero as a result of this loss, and if it is not a member of a proto, it dies and is removed from the simulation. If it \textit{is} a member of a proto, it withdraws the necessary amount from its proto's balance to remain at zero wealth. If the proto does not have sufficient funds to cover the loss, both the agent and the proto die and are removed from the simulation.
        \end{enumerate}
\end{enumerate}

Various statistical counters are updated as the program executes so that its behavior can be analyzed postmortem. The main simulation loop can also be invoked from a ``parameter sweep'' program which executes it multiple times over a range of parameter values, in order to determine how the model's behavior changes in response to key parameters.

% cite Bootstrap?
We chose Julia for its flexible type system, its speed of execution, its ease of programming, and the availability of useful packages, such as LightGraphs, Gadfly~\cite{Gadfly-2018} (plotting), and Bootstrap (confidence intervals).  Additionally, Julia easily allows the programmer to invoke code from R packages when necessary, as we did for Gini coefficient calculation with the R package DescTools~\cite{DescTools-2019}.


\section{Verification}

We first verify that the simulation's output matches obviously expected
results. Then, in the following section, we investigate aspects of its behavior
which are not computable analytically in order to discover important
consequences of the model.


\subsection{Agent and proto life history}

For a sensible range of parameter settings, the life history of a single
simulation run follows an expected pattern. Assuming a positive growth rate
(\textit{i.e.}, salary > metabolic rate), each agent's wealth rises unsteadily
during stage 1, until eventually the first pair of neighboring agents who each
reach the threshold form a proto-institution. Throughout stage 2, these agents
contribute all wealth in excess of the threshold to that proto, and so their
proto's balance rises unsteadily while their personal wealth remains at the
constant threshold. Meanwhile, the other agents also reach the threshold at
various points in time, and also form or join protos, until every non-isolate
(that is, every node with at least one neighbor) is a member of a proto. Stage
3 (the starvation period) then commences, with isolates drawing on a greater
personal wealth than the non-isolates. When an isolate reaches zero wealth, it
dies; when a non-isolate reaches zero, it draws from its (shared) proto balance
until that too, reaches zero, and it dies.

\begin{figure}[ht]
\centering
\includegraphics[width=\columnwidth]{figures/sampleLifeHistory.png}
\caption{A single run of the simulation, with $\lambda$=2. Each of 50 nodes is
given an initial wealth of $\sim\mathcal{U}(0,50)$ units, a regular income
distributed as $\sim\mathcal{N}(20,5)$, a metabolic rate of 5, and a proto
threshold of 65.} \label{fig:singleRun}
\end{figure}

This history can be seen in Figure~\ref{fig:singleRun}. The top plot depicts
agent wealth at every iteration, and the bottom plot shows the balances of the
protos at the same points in time. (No protos exist until stage 2, by
definition.) Note that the isolates (orange lines) never form protos, and
therefore begin the starvation stage with a higher personal wealth to draw
from.


\begin{figure}[ht]
\centering
\includegraphics[width=\columnwidth]{figures/sampleEffectiveHistory.png}
\caption{The same simulation run as Figure~\ref{fig:singleRun}, but this time
depicting each agent's \textit{effective} wealth (its personal wealth plus its
share of its proto's wealth, if any).}
\label{fig:effectiveWealthSingleRun}
\end{figure}

Figure~\ref{fig:effectiveWealthSingleRun} shows the same information in another
way: instead of plotting each agent's \textit{personal} wealth (as in the top
plot of Figure~\ref{fig:singleRun}), we show its \textbf{effective wealth},
defined as the sum of its personal wealth and its ``share'' of its proto's
wealth (if any). After all, contributions that a proto's members make to its
balance are available to those members in times of need; therefore, a fair
comparison between isolates and non-isolates should take this into account.
From the figure, it can be seen that isolates no longer have a systematic
advantage (as they appeared to in Figure~\ref{fig:singleRun}.)

We verified that changes to basic parameters all have the expected effect:
lowering the initial wealth delays the onset of stage 2; a higher $\sigma^2$
for the income distribution makes the lines less jagged; a higher metabolic
rate hastens extinction; \textit{etc.}

\subsection{Gini coefficient history}

% Will's paragraph goes here, references Figure~\ref{fig:giniHistory}.

\begin{figure}[hb]
\centering
\includegraphics[width=\columnwidth]{figures/giniHistory.png}
\caption{The simulated society's wealth inequality over time. (The same
simulation parameters were used as in Figures~\ref{fig:singleRun} and
\ref{fig:effectiveWealthSingleRun}, but this time with 500 agents.)}
\label{fig:giniHistory}
\end{figure}



\section{Analysis}

\subsection{Gini coefficient}

As mentioned in Section~\ref{giniKindaSucks}, the Gini coefficient is not the
ideal measure of inequality for our apocalyptic model. Nonetheless, it is
illustrative to see how it varies with respect to the ER $\lambda$ parameter.
Figure~\ref{fig:giniVsLambda} depicts the Gini computed \textit{at the onset
of stage 3} (before starvation) versus $\lambda$, and confirms that...
% Will, finish this paragraph.

\begin{figure}[hb]
\centering
\includegraphics[width=\columnwidth]{figures/giniVsLambda.png}
\caption{The average Gini coefficient, computed before the onset of starvation,
for various values of the ER $\lambda$ connectivity parameter.}
\label{fig:giniVsLambda}
\end{figure}

% Will's text and plot showing Gini coefficient (pre-stage-3) decreases with λ

\subsection{Life expectancy}

% Plots from 7/12 showing you best get in a proto if lambda and white noise 
%   are high


% Wealth histogram is steady ?


\section{Conclusion and Future Work}
This is one of the first phases of an ongoing project, and the next directions are too many to enumerate here. Still, we mention a few next steps that adhere to the spirit of our project. These directions all alleviate some of the obvious limitations of the current work. 

An obvious and important limitation of the current model is the lack of stability in wealth inequality in the population; neither the ER graph structure nor the Brownian noise stabilizes procured wealth. Therefore, Additional mechanisms are required to maintain the inequality.

While both the stochastic models and ER models are well understood, proto formation dynamics, as coagulation processes in continuous time, remain to be better understood, mathematically. Also, while the mathematical analysis of dynamics of discrete-time stochastic models on networks is well studied, diffusion processes on networks seem underexplored in the literature. 

Even with only a few parameters, the simulation results are challenging to visualize and interpret. As the number of parameters increases, as expected in future models, ideas from design and analysis of (computer) experiments may be required. Also, our exploration of the time to death distributions of the population show non-trivial structure. More careful statistical tests that go beyond the mean analysis presented here are required to tease out the necessary and sufficient conditions for agents in protos (non-isolates) to consistently outperform the isolates.   

The models used for our simulation are at best \textit{stylized} models of real world social and economic systems, especially in anthropology~\cite{white2011kinship} and historical and urban sociology~\cite{sampson2012great,katz2013people}. Research in economic anthropology~\cite{koster2019,koster2014,koster2015,bogerhoff2015,smith2019,nolin2012,power2018cooperation,power2018} suggests complex food and economic resource sharing rituals among members of various communities. Such resource sharing social networks do not look like any of the \textit{textbook} models. Extending our analysis to more expressive network models like exponential random graph models, stochastic block models and latent space models is an important research direction. This alongside the use of empirically observed cooperation and coordination protocols have potential in making our models better calibrated with real world systems.   

Despite their simplicity, models like the ones constructed here have several advantages.  As ACE models, they offer insights about economic systems in which the majority of the assumptions of neo-classical economics like perfectly mixed agents~\cite{wilhite} and presence of equilibrium~\cite{arthur} do not hold. As AS models~\cite{hedstrom2011oxford}, they offer an approach that adds models of social mechanisms to CSS models in a graded manner.  

\begin{acks}
We thank Milton Friesen and Srikanth Mudigonda, the authors of ~\cite{srimil}, for providing the code for resource pooling; Thomas Davies for contributions to the code base; and Olufemi Olaba for modeling ideas. This work is part of a larger project undertaken by ABMSPECSIG\footnote{https://github.com/WheezePuppet/specstar} team. The authors wish to thank their respective home institutions where this work was done.
\end{acks}


% this line is for Stephen's purposes only:
% vim:textwidth=99999
